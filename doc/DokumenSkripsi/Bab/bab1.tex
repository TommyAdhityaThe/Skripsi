\chapter{Pendahuluan}
\label{chap: pendahuluan}

\section{Latar Belakang}
\label{sec:latarBelakang}
Kemacetan merupakan salah satu masalah yang umumnya ditemukan di kota-kota besar. Kemacetan telah menghabiskan banyak sekali waktu produktivitas orang-orang yang tinggal di kota besar. Kemacetan dapat terjadi karena banyak faktor. Salah satu faktor kemacetan adalah karena jumlah kendaraan yang ada dalam suatu kota tidak sebanding dengan kapasitas jalan yang ada pada kota tersebut.

Bandung merupakan salah satu kota besar di mana kemacetan terjadi hampir setiap hari terutama pada saat waktu menjelang berangkat dan pulang kerja. Kemacetan terjadi dikarenakan setiap individu di Bandung umumnya memiliki jam kerja yang sama. Jam kerja yang sama membuat ledakan kendaraan terjadi di jalanan pada saat-saat tertentu. Dikarenakan kapasitas jalanan tidak dapat menampung banyaknya kendaraan yang ada maka terjadilah kemacetan.

Salah satu solusi untuk mengatasi kemacetan adalah penggunaan fasilitas angkutan umum. Angkutan umum akan mengurangi penggunaan kendaraan pribadi secara drastis. Faktanya, di Bandung orang-orang masih jarang menggunakan angkutan umum. Orang takut dan bingung menggunakan angkutan umum dikarenakan kendaraan umum seperti angkot memiliki rute jalannya masing-masing sesuai kode angkot. Terkadang perlu digunakan lebih dari 1 angkot untuk sampai ke tujuan. Bandung juga merupakan kota yang banyak dihuni oleh kaum-kaum pelajar(mahasiswa) asal luar Bandung. Otomatis banyak penghuni Bandung adalah orang-orang yang masih buta akan jalan apalagi rute angkutan umum yang ada di Bandung.

KIRI merupakan sebuah solusi untuk permasalahan kebutaan rute angkutan umum masyarakat yang tinggal di Bandung. KIRI memberikan layanan berupa pengetahuan akan rute angkutan umum yang harus ditempuh oleh seseorang untuk mencapai tujuannya dari tempat orang tersebut berasal. Dengan adanya KIRI orang-orang akan menjadi lebih berani dan siap untuk menggunakan kendaraan umum dalam bepergian baik orang asli Bandung maupun orang dari luar Bandung. Permasalahan kemacetan di Bandung diharapkan dapat berkurang atau bahkan terselesaikan dengan adanya bantuan dari KIRI.

KIRI terbagi ke dalam 2 bagian penting, yaitu: \textit{frontend} dan \textit{dashboard}. Pada bagian \textit{frontend} pada KIRI merupakan bagian \textit{user interface} dan juga sekaligus merupakan halaman utama KIRI, yang dapat diakses melalui alamat \href{http://kiri.travel}{http://kiri.travel}. Pada bagian \textit{dashboard} pada KIRI merupakan bagian yang digunakan untuk tim \textit{developer} untuk memperbaiki ataupun melakukan pengembangan sistem KIRI, yang dapat diakses melalui alamat \href{https://dev.kiri.travel/bukitjarian}{https://dev.kiri.travel/bukitjarian}. KIRI menggunakan bahasa pemrograman PHP untuk membuat bagian \textit{dashboard server side}.

KIRI merupakan proyek besar yang akan terus berkembang seiring dengan perkembangan waktu dikarenakan kebutuhan masyarakat akan rute angkutan umum akan selalu ada dan terus berkembang. Bahasa PHP kurang cocok untuk proyek skala besar seperti KIRI. Tidak ada deklarasi dan tipe variabel yang jelas pada bahasa PHP. Setiap programer PHP memiliki ciri khasnya sendiri dalam membangun program. Untuk setiap programer baru yang ingin mengembangkan sistem perlu mempelajari sistem secara detail terlebih dahulu agar mampu mengerti konsep desain program yang ada. Java merupakan bahasa yang umum dan lebih terstruktur. Play Framework membantu implementasi Java untuk pembuatan \textit{website}. Play Framework menggunakan prinsip MVC sehingga setiap programer sudah langsung mengetahui desain program yang ada. Play Framework membuat setiap programer memiliki konsep dan prinsip desain program yang sama antara yang satu dan yang lainnya, sehingga mempermudah programer untuk melakukan pengembangan sistem.

Berdasarkan permasalahan-permasalahan yang telah peneliti jelaskan, peneliti tertarik untuk melakukan perombakan sistem bagian \textit{dashboard} KIRI(\href{https://dev.kiri.travel/bukitjarian}{https://dev.kiri.travel/bukitjarian}). Peneliti tertarik untuk melakukan perombakan bahasa yang diguanakan, dari yang semula menggunakan bahasa PHP menjadi menggunakan Java dengan menggunakan Play Framework. Untuk itu dibuatlah penelitian ``Porting PHP menjadi Java dan Play Framework. Studi Kasus: KIRI Dashboard Server Side'' guna menyelesaikan masalah-masalah yang ada.

\section{Rumusan Masalah}
\label{sec:rumusan_masalah}
	Berikut adalah susunan permasalahan yang akan dibahas pada penelitian ini:
	\begin{enumerate}
		\item Bagaimana rancangan dan cara kerja sistem \textit{dashboard server side}  KIRI versi lama dengan menggunakan bahasa pemrograman PHP?
		\item Apa saja kelebihan dan kekurangan rancangan sistem \textit{dashboard server side}  KIRI versi lama dengan menggunakan bahasa pemrograman PHP?
		\item Bagaimana rancangan dan cara kerja sistem \textit{dashboard server side}  KIRI versi baru dengan menggunakan bahasa pemrograman Java dengan memanfaatkan Play Framework?
		\item Apa saja kelebihan dan kekurangan rancangan sistem \textit{dashboard server side}  KIRI versi baru dengan menggunakan bahasa pemrograman Java dengan memanfaatkan Play Framework?
	\end{enumerate}

\section{Tujuan}
\label{sec:tujuan}
	Berdasarkan rumusan masalah yang telah dibuat, maka tujuan penelitian ini dijelaskan ke dalam poin-poin sebagai berikut:
	\begin{enumerate}
		\item Mengetahui rancangan dan cara kerja sistem \textit{dashboard server side}  KIRI versi lama dengan menggunakan bahasa pemrograman PHP,
		\item Mengetahui kelebihan dan kekurangan rancangan sistem \textit{dashboard server side}  KIRI versi lama dengan menggunakan bahasa pemrograman PHP,
		\item Mengetahui rancangan dan cara kerja sistem \textit{dashboard server side}  KIRI versi baru dengan menggunakan bahasa pemrograman Java dengan memanfaatkan Play Framework,
		\item Mengetahui kelebihan dan kekurangan rancangan sistem \textit{dashboard server side}  KIRI versi baru dengan menggunakan bahasa pemrograman Java dengan memanfaatkan Play Framework.
	\end{enumerate}
