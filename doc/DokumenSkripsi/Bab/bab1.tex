\chapter{Pendahuluan}
\label{chap:pendahuluan}

\section{Latar Belakang}
\label{sec:latar_belakang}
KIRI\cite{kiritravel} merupakan situs web untuk membantu pengguna menemukan rute transportasi umum ke tempat tujuannya. Dengan memasukkan lokasi awal serta lokasi tujuan pengguna tersebut, situs web KIRI akan memberikan langkah-langkah (contoh: berjalan sejauh berapa meter, menggunakan angkot, dll) tercepat untuk sampai ke lokasi tujuan. Situs web KIRI telah hadir di berbagai kota seperti Bandung, Depok, Jakarta, Surabaya dan Malang.

KIRI \textit{Dashboard}\cite{devkiritravel} adalah bagian dari situs web KIRI. KIRI \textit{Dashboard} berfungsi sebagai pengatur proses CRUD (\textit{Create, Read, Update} dan \textit{Delete}) daftar rute yang terdapat dalam \textit{database} situs web KIRI. KIRI \textit{Dashboard Server Side} menggunakan bahasa PHP dalam pembuatannya\cite{kiridashboard}. Bahasa PHP kurang cocok untuk proyek besar seperti \textit{dashboard}. Salah satu penyebab bahasa PHP kurang cocok adalah karena tidak ada deklarasi dan tipe variabel dalam penggunaan bahasa PHP.

Java merupakan bahasa pemrograman yang umum digunakan oleh banyak orang. Selain umum digunakan, Java juga merupakan bahasa pemrograman yang lebih terstruktur dibandingkan PHP. Adanya deklarasi dan tipe variabel pada Java membuat setiap variabel memiliki kegunaan yang lebih jelas dan mudah dimengerti. Play Framework merupakan \textit{framework} yang membantu implementasi Java dalam pembuatan suatu situs web. Play Framework juga cocok untuk proyek skala besar karena arsitekturnya sudah menggunakan konsep MVC (\textit{Model View Controller})\cite{playforjava}.

Berdasarkan ditemukannya kekurangan-kekurangan pada KIRI \textit{Dashboard Server Side} seperti yang telah dijelaskan, maka solusi untuk mengatasi kekurangan tersebut adalah dibuatlah penelitian ini untuk mengubah KIRI \textit{Dashboard Server Side} yang semula dalam bahasa PHP menjadi bahasa Java dengan menggunakan Play Framework.

\section{Rumusan Masalah}
\label{sec:rumusan_masalah}
Berikut adalah susunan permasalahan yang akan dibahas pada penelitian ini:
	\begin{enumerate}
		\item Bagaimana isi kode KIRI \textit{Dashboard Server Side} dan apa saja kekurangan yang ada di dalamnya?
		\item Bagaimana cara kerja Play Framework berbasis MVC?
		\item Bagaimana melakukan \textit{porting} KIRI \textit{Dashboard Server Side} yang semula dalam bahasa PHP menjadi bahasa
Java dengan menggunakan Play Framework?
	\end{enumerate}
	
\section{Tujuan}
\label{sec:tujuan}
Berdasarkan rumusan masalah yang telah dibuat, maka tujuan skripsi ini dijelaskan ke dalam poin-poin sebagai berikut:
	\begin{enumerate}
		\item Mengetahui isi kode KIRI \textit{Dashboard Server Side} dan kekurangan-kekurangan yang ada di dalamnya.
		\item Mengetahui cara kerja Play Framework berbasis MVC.
		\item Melakukan \textit{porting} KIRI \textit{Dashboard Server Side} yang semula dalam bahasa PHP menjadi bahasa
Java dengan menggunakan Play Framework
	\end{enumerate}
	
\section{Batasan Masalah}
\label{sec:batasan_masalah}
Skripsi ini dibuat berdasarkan batasan-batasan sebagai berikut:
	\begin{enumerate}
		\item Play Framework yang digunakan selama penulisan skripsi ini adalah versi 2.4.3.
		\item \textit{Porting} Kode KIRI \textit{Dashboard Server Side} yang dilakukan adalah berdasarkan versi terbaru dari Github dengan \textit{username}: ``pascalalfadian''\cite{kiridashboard}. 
	\end{enumerate}
	
\section{Metode Penelitian}
\label{sec:metode_penelitian}
Berikut adalah metode penelitian yang digunakan dalam pembuatan skripsi ini:
	\begin{enumerate}
		\item Melakukan studi literatur mengenai kode KIRI \textit{Dashboard Server Side}, MySQL Spatial Extensions dan Play Framework.
		\item Menganalisis teori-teori untuk membangun KIRI \textit{Dashboard Server Side} dalam bahasa Java dengan menggunakan Play Framework.
		\item Merancang KIRI \textit{Dashboard Server Side} dalam bahasa Java dengan menggunakan Play Framework.
		\item Melakukan \textit{porting} kode situs web KIRI \textit{Dashboard Server Side} menjadi Java dengan menggunakan Play Framework.
		\item Melakukan pengujian terhadap fitur-fitur yang sudah dibuat.
	\end{enumerate}

\section{Sistematika Penulisan}
\label{sec:sistematika_penulisan}
Setiap bab dalam penulisan ini memiliki sistematika yang dijelaskan ke dalam poin-poin sebagai berikut:
	\begin{enumerate}
		\item Bab 1: Pendahuluan, yaitu membahas mengenai gambaran umum skripsi ini yang berisi tentang latar belakang, rumusan masalah, tujuan, batasan masalah, metode penelitian dan sistematika penulisan.
		\item Bab 2: Dasar Teori, yaitu membahas mengenai teori-teori yang mendukung berjalannya skripsi ini yang berisi tentang kode KIRI \textit{Dashboard Server Side}, penggunaan MySQL Spatial Extensions, dan Play Framework.
		\item Bab 3: Analisis, yaitu membahas mengenai analisa masalah yang berisi tentang kode KIRI \textit{Dashboard Server Side} beserta kekurangan-kekurangannya dan cara melakukan \textit{porting} bahasa PHP menjadi Java/Play Framework kode KIRI \textit{Dashboard Server Side}.
	\end{enumerate}