\chapter{Pendahuluan}
\label{chap:pendahuluan}

\section{Latar Belakang}
\label{sec:latar_belakang}

KIRI (\url{http://kiri.travel}) adalah sebuah situs yang dikelola oleh PT.
Kirana Sistem Transportasi, yang menyediakan layanan navigasi dari satu titik
ke titik lain memanfaatkan transportasi publik. Layanan ini diberikan secara
gratis kepada pengunjung situs / pengguna aplikasi bergerak mereka. Untuk
mendukung layanan tersebut, tim KIRI melakukan kurasi rute angkot secara
mandiri di Bandung dengan mencatat perjalanan setiap trayek menggunakan
peralatan GPS (\textit{Global Positioning System}). Pada perkembangannya, KIRI
melakukan ekspansi ke beberapa kota lainnya termasuk Jakarta. Hanya saja,
karena keterbatasan sumberdaya data di Jakarta dimasukkan berbekal data yang
tersedia di internet, tanpa validasi di lapangan. Di sisi lain, ada beberapa
pihak yang tertarik akan layanan KIRI di Bandung dan berniat untuk mendapatkan
layanan serupa di kota mereka.

Situs \url{https://angkot.web.id} (selanjutnya disebut angkot.web.id saja)
merupakan sebuah situs yang dikembangkan oleh Fajran Iman Rusadi yang berbasis
di Belanda. Situs ini memungkinkan pengguna publik untuk melihat, memasukkan, 
atau memperbaiki data rute angkot di Indonesia (dengan kata lain,
\textit{crowdsourcing}). Layanan ini juga diberikan secara gratis.

Sampai saat ini, KIRI serta angkot.web.id merupakan dua buah situs yang
terpisah dan bekerja secara independen.

\section{Rumusan Masalah}
\label{sec:rumusan_masalah}
Dari latar belakang yang sudah dijelaskan, peneliti bermaksud untuk mengintegrasikan
data yang dimiliki kedua situs web tersebut. Integrasi tersebut dirumuskan ke dalam
masalah-masalah berikut:
\begin{itemize}
	\item Bagaimana mekanisme penarikan data oleh KIRI terhadap angkot.web.id
		secara otomatis?
	\item Bagaimana memisahkan data yang dimiliki oleh KIRI sendiri dengan data
		yang ditarik dari angkot.web.id?
	\item Bagaimana mengoptimasi protokol yang digunakan, sehingga kebutuhan
		\textit{bandwidth} dapat dihemat?
	\item Bagaimana respon pengguna KIRI terhadap fitur yang diimplementasikan?
\end{itemize}

\section{Tujuan}
\label{sec:tujuan}
Berdasarkan rumusan masalah yang sudah dijabarkan, maka didefinisikan tujuan-tujuan
berikut:
\begin{itemize}
	\item Mengimplementasikan mekanisme penarikan data otomatis oleh KIRI
		terhadap angkot.web.id.
	\item Mengimplementasikan pemisahan data antara rute milik KIRI dengan data
		yang ditarik dari angkot.web.id.
	\item Mengoptimasi protokol yang digunakan, sehingga kebutuhan
		\textit{bandwidth} dapat dihemat.
	\item Mempelajari respon pengguna KIRI terhadap fitur yang diimplementasikan.
\end{itemize}

\section{Batasan Masalah}
\label{sec:batasan_masalah}
Penelitian ini memiliki batasan-batasan seperti berikut:
\begin{itemize}
	\item Penelitian dilakukan untuk rute angkot kota Bandung saja.
	\item Integrasi otomatis akan dilakukan secara berkala
		(tidak \textit{realtime}).
	\item Dengan alasan kerahasiaan, mesin navigasi KIRI dan angkot.web.id
		hanya dijelaskan pada bagian-bagian yang terkait dengan penelitian
		ini saja.
\end{itemize}

\section{Metode Penelitian}
\label{sec:metode_penelitian}
Dalam penelitian ini, akan dilakukan langkah-langkah berikut:
\begin{itemize}
	\item Melakukan studi terhadap mesin navigasi KIRI, protokol angkot.web.id,
		serta teori-teori lain yang mendukung kedua hal tersebut.
	\item Melakukan analisis untuk menemukan hal yang dapat dilakukan untuk
		mengintegrasikan data kedua situs tersebut.
	\item Melakukan perancangan untuk implementasi integrasi kedua sistem
		tersebut.
	\item Melakukan implementasi dari rancangan yang sudah dilakukan.
	\item Melakukan publikasi terhadap pengguna KIRI sehingga mereka dapat
		menguji hasil implementasi tersebut.
	\item Menarik kesimpulan atas hasil yang telah dilaksanakan.
\end{itemize}

\section{Sistematika Penulisan}
\label{sec:sistematika_penulisan}
Berikut adalah sistematika penulisan dari dokumen ini:
\begin{itemize}
	\item Bab 1 membahas latar belakang, rumusan masalah, tujuan penulisan,
		batasan-batasan, serta metode yang digunakan pada penelitian ini.
	\item Bab 2 membahas teori-teori yang digunakan dalam penelitian ini.
	\item Bab 3 membahas analisis yang dilakukan terhadap teori yang telah
		dijabarkan pada bab 2.
	\item Bab 4 membahas perancangan yang dilakukan sebelum
		mengimplementasikan integrasi yang dimaksud.
	\item Bab 5 membahas implementasi serta pengujian dari integrasi yang
		telah dilakukan.
	\item Bab 6 membahas kesimpulan dari keseluruhan penelitian ini, serta
		saran-saran yang dapat diberikan untuk penelitian berikutnya.
\end{itemize}