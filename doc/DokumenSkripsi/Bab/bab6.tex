\chapter{Kesimpulan dan Saran}
\label{chap:kesimpulansaran}

\section{Kesimpulan}
\label{sec:kesimpulan}
Berdasarkan hasil penelitian yang dilakukan, didapatkan kesimpulan-kesimpulan sebagai berikut:
\begin{enumerate}
	\item Kode KIRI \textit{Dashboard server side} dapat dibagi menjadi 16 yang masing-masing melayani sebuah permintaan tertentu untuk bagian tampilan sistem KIRI, yaitu: pemeriksaan \textit{login}, \textit{login}, \textit{logout}, menambahkan rute, mengubah rute, melihat daftar rute, melihat informasi rute secara detail, menghapus data geografis suatu rute, impor data KML, menghapus rute, melihat daftar API \textit{keys}, menambahkan API \textit{key}, mengubah API \textit{key}, \textit{register}, melihat data pribadi pengguna, dan mengubah data pribadi pengguna.
	\item Telah berhasil melakukan \textit{porting} kode KIRI \textit{Dashboard server side} yang semula dalam bahasa PHP menjadi bahasa Java dengan menggunakan Play Framework. \textit{Porting} dilakukan dengan memodelkan 16 bagian kode KIRI \textit{Dashboard server side} menjadi \textit{models} dan \textit{controllers} pada Play Framework. Bagian tampilan kode KIRI \textit{Dashboard} dapat disalin apa adanya dalam Play Framework untuk mencoba seluruh fitur KIRI \textit{Dashboard server side}.
	\item Berdasarkan hasil pengujian eksperimental, waktu eksekusi fitur-fitur KIRI \textit{Dashboard server side} yang dibangun dengan Play Framework umumnya lebih cepat dibandingkan dengan KIRI \textit{Dashboard server side} yang dibangun dengan PHP, kecuali pada fitur \textit{register}, \textit{login}, dan mengubah data pribadi pengguna.
\end{enumerate}

\section{Saran}
\label{sec:saran}
Berdasarkan hasil penelitian yang dilakukan, berikut adalah beberapa saran untuk pengembangan:
\begin{enumerate}
	\item Melakukan analisa protokol internet yang digunakan pada metode pengiriman \textit{email} fitur \textit{register} sistem KIRI \textit{Dashboard server side} yang dibangun dengan PHP dan KIRI \textit{Dashboard server side} yang dibangun dengan Play Framework dengan menggunakan perangkat analisa protokol internet (contoh: Wireshark).
	\item Melakukan pengujian eksperimental (terhadap waktu eksekusi) ulang terhadap 3 fitur KIRI \textit{Dashboard server side}, yaitu: \textit{login}, \textit{register}, dan mengubah data pribadi pengguna dengan ketentuan sebagai berikut:
	\begin{itemize}
		\item Mengubah nilai kompleksitas proses \textit{hashing} pada fitur \textit{login} dan mengubah data pribadi pengguna sistem KIRI \textit{Dashboard server side} yang dibangun dengan Play Framework yang semula adalah 10 menjadi 8.
		\item Mengubah protokol internet yang digunakan pada fitur \textit{register} sistem KIRI \textit{Dashboard server side} yang dibangun dengan Play Framework menjadi sama dengan protokol internet yang digunakan pada fitur \textit{register} sistem KIRI \textit{Dashboard server side} yang dibangun dengan PHP.
	\end{itemize}
	\item Menambahkan fitur untuk menghapus API \textit{keys}.
\end{enumerate}