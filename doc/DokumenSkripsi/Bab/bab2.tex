\chapter{Dasar Teori}
\label{chap:dasar_teori}

\section{MySQL Spatial Extensions}
\label{sec:mysql_spatial_ex}
	Suatu \textit{geographic feature}\cite{mysqlspatial} adalah sesuatu yang ada di bumi yang memiliki lokasi sebagai penunjuk letak keberadaannya. Geometri adalah cabang ilmu matematika yang digunakan untuk memodelkan suatu \textit{geographic feature}. Dengan geometri, suatu \textit{geographic feature} dapat dinyatakan sebagai sebuah titik, garis, ruang, ataupun bentuk lainnya. Suatu ``\textit{feature}'' yang dimaksud dalam istilah \textit{geographic feature} dapat berupa:
	\begin{enumerate}
		\item \textbf{\textit{An entity}}, contohnya adalah gunung, kolam, kota, dll.
		\item \textbf{\textit{A space}}, contohnya adalah daerah, cuaca, dll.
		\item \textbf{\textit{A definable location}}, contohnya adalah persimpangan jalan, yaitu suatu tempat khusus dimana terdapat 2 buah jalan yang saling berpotongan.
	\end{enumerate}
	
	MySQL adalah salah satu perangkat lunak yang digunakan untuk mengatur data-data (\textit{database}) suatu situs web. Bentuk MySQL adalah sekumpulan tabel-tabel yang umumnya memiliki hubungan antar satu dengan yang lainnya. Setiap tabel pada MySQL memiliki kolom dan baris. Kolom pada MySQL menyatakan daftar jenis baris yang ingin dibuat dan baris menyatakan banyaknya data yang ada dalam tabel.
	
	Penamaan suatu kolom dalam MySQL membutuhkan penentuan jenis tipe data yang akan digunakan dalam kolom tersebut. Dalam MySQL terdapat tipe-tipe data yang umum digunakan seperti \textit{Varchar} untuk menyimpan karakter atau kata, \textit{Int} untuk menyimpan angka, \textit{Boolean} untuk menyimpan nilai ``\texttt{true}'' atau ``\texttt{false}'', dan tipe data lainnya. MySQL Spatial Extensions adalah perluasan dari tipe-tipe data yang disediakan MySQL untuk menyatakan nilai geometri dari suatu \textit{geographic feature}.
	
\subsection{Tipe Data Spatial}
\label{sec:tipe_data_spatial}
Berdasarkan kemampuan penyimpanan nilai geometri\cite{spatialtipedata}, tipe data \textit{spatial} dapat dikelompokan ke dalam 2 jenis:
\begin{enumerate}
	\item Tipe data yang hanya dapat menyimpan sebuah nilai geometri saja, yaitu:
	\begin{itemize}
		\item \textbf{\textit{Geometry}}
		\item \textbf{\textit{Point}}
		\item \textbf{\textit{LineString}}
		\item \textbf{\textit{Polygon}}
	\end{itemize}
	\item Tipe data yang dapat menyimpan sekumpulan nilai geometri, yaitu:
	\begin{itemize}
		\item \textbf{\textit{MultiPoint}}
		\item \textbf{\textit{MultiLineString}}
		\item \textbf{\textit{MultiPolygon}}
		\item \textbf{\textit{GeometryCollection}}
	\end{itemize}
\end{enumerate}

\subsubsection{\textit{Point}}
\label{sec:point}
\textit{Point} adalah nilai geometri yang merepresentasikan sebuah lokasi ke dalam suatu koordinat\cite{spatialtipedatapoint}. Koordinat pada \textit{Point} terdiri dari nilai X dan Y dimana X merepresentasikan letak lokasi dalam garis bujur dan Y merepresentasikan letak lokasi dalam garis lintang. \textit{Point} tidak memiliki dimensi maupun nilai batasan. Contoh representasi \textit{Point} adalah Universitas Katolik Parahyangan direpresentasikan dalam koordinat X=-6.874735 dan Y=107.6049079 pada skala tertentu (gambar \ref{fig:2_UNPAR}).

\begin{figure}[htbp]
	\centering
		\includegraphics[scale=0.4]{Gambar/2_UNPAR.JPG}
	\caption{Koordinat lokasi Universitas Katolik Parahyangan (\textbf{lingkaran merah})}
	\label{fig:2_UNPAR}
\end{figure}

\subsubsection{\textit{LineString}}
\label{sec:linestring}
\textit{LineString} adalah garis yang terbentuk dari sekumpulan \textit{Point}\cite{spatialtipedatalinestring}. Dalam peta dunia, \textit{LineString} dapat merepresentasikan sebuah sungai dan dalam peta perkotaan, \textit{LineString} dapat merepresentasikan sebuah jalan (contoh: gambar \ref{fig:2_linestring}). Karena \textit{LineString} merupakan sekumpulan \textit{Point}, maka \textit{LineString} menyimpan sekumpulan koordinat dimana setiap koordinat ($X_{1}$..$X_{n}$ dan $Y_{1}$..$Y_{n}$, dimana n menyatakan banyaknya \textit{Point} dalam \textit{LineString}) terhubung oleh garis dengan koordinat selanjutnya. Contohnya: misal terdapat sebuah \textit{LineString} yang mengandung 3 buah \textit{Point}, maka terdapat garis yang menghubungkan \textit{Point} pertama dengan \textit{Point} kedua dan \textit{Point} kedua dengan \textit{Point} ketiga.

\begin{figure}[htbp]
	\centering
		\includegraphics[scale=0.4]{Gambar/2_LineString.JPG}
	\caption{Rute (jalan) yang harus ditempuh dari Cawan Kitchen untuk sampai ke Universitas Katolik Parahyangan direpresentasikan dengan \textit{LineString} (\textbf{garis hijau} dan \textbf{garis merah})}
	\label{fig:2_linestring}
\end{figure}

\section{Play Framework}
\label{sec:play_framework}
Play Framework adalah sekumpulan kerangka kode yang dapat digunakan untuk membangun suatu situs web. Play Framework tidak hanya menggunakan bahasa Java dalam pembuatannya. Bahasa Scala juga digunakan Play Framework dalam beberapa bagian seperti bagian \textit{view} dan \textit{route}\cite{playforjava}. Play Framework menggunakan konsep MVC (\textit{Model} \textit{View} \textit{Controller}) sebagai pola arsitekturnya. Konsep MVC pada suatu kode membuat kode mudah dikembangkan baik secara tampilan maupun pengembangan fitur-fiturnya. Ketika \textit{server} Play Framework dijalankan, secara \textit{default} dapat diakses melalui ``localhost:9000''.

\subsection{Struktur Aplikasi}
\label{sec:struktur_aplikasi}
Ketika Play Framework pertama kali ter-\textit{install} pada komputer, Play Framework menyediakan \textit{default} direktori dengan struktur minimal (gambar \ref{fig:2_strukturplay}). Berikut adalah penjelasan struktur minimal Play Framework:
\begin{enumerate}
	\item \textit{Folder} ``app'' merupakan \textit{folder} yang berisi mengenai pola arsitektur yang dimiliki Play Framework, yaitu ``models'' (tidak dibuat secara \textit{default}), ``views'', dan ``controllers'' yang akan dijelaskan lebih lanjut pada subbab selanjutnya (subbab \textit{Models}: \ref{sec:models}, subbab \textit{Views}: \ref{sec:views}, dan subbab \textit{Controllers}: \ref{sec:controllers}).
	\item \textit{Folder} ``conf'' berisi mengenai \textit{file} ``application.conf'' yang menyimpan pengaturan-pengaturan seperti kumpulan \textit{log}, koneksi ke \textit{database}, jenis \textit{port} tempat \textit{server} bekerja, dll. \textit{Folder} ``conf'' juga berisi \textit{file} ``routes'' yang mengatur bagaimana HTTP \textit{requests} nantinya akan diproses lebih lanjut yang akan dijelaskan pada subbab selanjutnya (subbab \ref{sec:routes}).
	\item \textit{Folder} ``project'' terdapat \textit{file} ``build.properties'' dan ``plugins.sbt'', \textit{file} tersebut mendeskripsikan versi Play dan SBT yang digunakan pada aplikasi.
	\item \textit{Folder} ``public'' merupakan \textit{folder} yang menyimpan data-data seperti gambar (\textit{folder} ``images''), kumpulan Javascript yang digunakan (\textit{folder} ``javascripts'', secara \textit{default} berisikan \textit{file} ``jquery-1.9.0.min.js'') dan data-data CSS (folder ``stylesheets'').
	\item \textit{File} ``build.sbt'' mengatur \textit{dependencies} yang dibutuhkan dalam pembuatan aplikasi.
	\item Terakhir adalah \textit{folder} ``test'' yang merupakan salah satu kelebihan dari Play Framework, bagian ini berisikan \textit{file} ``Application.test'' dan ``Integration.test'' yang dapat digunakan untuk melakukan serangkaian \textit{testing} yang diinginkan terhadap aplikasi.
\end{enumerate}
     

\begin{figure}[htbp]
	\centering
		\includegraphics[scale=0.7]{Gambar/2_strukturplay.JPG}
	\caption{Struktur minimal Play Framework}
	\label{fig:2_strukturplay}
\end{figure}

\subsection{\textit{Routes}}
\label{sec:routes}
\textit{Routes} adalah \textit{file} yang mengatur pemetaan dari HTTP URLs menuju kode aplikasi (dalam hal ini menuju ke \textit{controllers}). Secara \textit{default}, \textit{routes} berisikan kode seperti pada gambar \ref{fig:2_routes1}. Kode \textit{default} pada \textit{routes} tersebut dapat memetakan permintaan URL \textit{index} standar seperti ``localhost:9000'' ketika \textit{server} Play Framework sudah dijalankan.

\begin{figure}[htbp]
	\centering
		\includegraphics[scale=0.8]{Gambar/2_routes1.JPG}
	\caption{Isi kode \textit{file} ``routes''\cite{playforjava}}
	\label{fig:2_routes1}
\end{figure}

Struktur \textit{routes} terdiri dari 3 bagian (gambar \ref{fig:2_routes2}), yaitu HTTP \textit{method}, URL \textit{path}, dan \textit{action method}. Struktur \textit{routes} seperti yang dijelaskan pada gambar \ref{fig:2_routes2} juga sekaligus menjadi struktur minimal yang harus ada agar \textit{routes} dapat memetakan suatu HTTP URLs. HTTP \textit{method} berisikan protokol yang ingin dilakukan terhadap suatu HTTP \textit{request}. HTTP \textit{method} dapat berupa ``\texttt{GET}'', ``\texttt{POST}'', ``\texttt{DELETE}'', ``\texttt{PATCH}'', ``\texttt{HEAD}'' atau ``\texttt{PUT}''\cite{playframeworkrouting1}. URL \textit{path} merupakan direktori yang ingin dituju dalam \textit{server} aplikasi. URL \textit{path} dimulai dengan tanda ``/'' dan diikuti dengan nama direktori yang ingin dituju. Terakhir, \textit{action method} merupakan pemilihan kelas \textit{controller} yang ingin dituju. Struktur \textit{action method} terdiri dari 3 bagian (dipisahkan dengan karakter ``.''), yaitu pemilihan \textit{package} ``controllers'' yang ingin dituju, bagian kedua adalah pemilihan kelas ``controllers'' yang dipilih (contohnya: ``Products'' pada gambar \ref{fig:2_routes2}), dan terakhir adalah pemilihan \textit{method} yang ada pada kelas ``controllers'' yang dipilih (contohnya: ``list()'').

\begin{figure}[htbp]
	\centering
		\includegraphics[scale=0.8]{Gambar/2_routes2.JPG}
	\caption{Struktur kode \textit{file} ``routes''\cite{playforjava}}
	\label{fig:2_routes2}
\end{figure}

URL \textit{path} dan \textit{action method} pada \textit{routes} juga dapat berisi sebuah nilai variabel (gambar \ref{fig:2_routes3}). Penulisan sebuah variabel pada URL \textit{path} dimulai dengan tanda ``:'' lalu diikuti dengan nama variabel yang diinginkan, contohnya: ``\texttt{:id}''. Ketika menggunakan variabel pada URL \textit{path}, pada \textit{action method} perlu ditambahkan deklarasi variabel yang ditaruh di dalam bagian \textit{method} yang dipilih. Cara penulisan deklarasi variabel pada \textit{action method} adalah dimulai dengan nama variabel, lalu diikuti karakter ``:'', dan diakhiri dengan tipe variabel yang diinginkan. Contoh penulisan deklarasi variabel di dalam \textit{method} suatu kelas pada bagian \textit{action method}: ``\texttt{id: Long}'' seperti dijelaskan pada gambar \ref{fig:2_routes3}. 

\begin{figure}[htbp]
	\centering
		\includegraphics[scale=0.8]{Gambar/2_routes3.JPG}
	\caption{Struktur kode \textit{file} ``routes'' dengan variabel\cite{playframeworkrouting2} (\textbf{lingkaran merah})}
	\label{fig:2_routes3}
\end{figure}

\subsection{\textit{Models}}
\label{sec:models}
Fungsi \textit{models} pada Play Framework sama seperti fungsi \textit{models} pada pola arsitektur MVC secara umum, yaitu untuk memanipulasi dan menyimpan data. Secara \textit{default}, \textit{models} tidak dibuat oleh struktur minimal Play Framework (gambar \ref{fig:2_strukturplay}). Untuk itu perlu menambahkan \textit{models} secara manual ke dalam struktur Play Framework. Langkah yang dilakukan untuk menambahkan \textit{models} ke dalam Play Framework adalah:
\begin{enumerate}
	\item Menambahkan folder ``models'' ke dalam folder ``app'',
	\item Menambahkan file dengan format ``.java'' ke dalam folder ``models''.
\end{enumerate}

Tidak ada aturan khusus yang diharuskan dalam penulisan kode dalam kelas \textit{models}. Selama kelas \textit{models} yang dibuat memenuhi aturan bahasa Java, maka \textit{models} dapat dieksekusi oleh \textit{server} Play Framework.

\subsection{\textit{Views}}
\label{sec:views}
Fungsi \textit{views} pada Play Framework adalah mengatur tampilan yang ingin ditampilkan di layar. \textit{Views} menggunakan bahasa HTML dan Scala. Bahasa Scala pada \textit{views} berfungsi sebagai penerima parameter yang dikirimkan dari kelas \textit{models} dimana antara \textit{models} dan \textit{views} dihubungkan oleh \textit{controllers}. Penamaan \textit{file} di dalam folder \textit{views} (gambar \ref{fig:2_strukturplay}) harus dengan format sebagai berikut, ``namaFile.scala.html''.

\begin{figure}[htbp]
	\centering
		\includegraphics[scale=0.8]{Gambar/2_views.JPG}
	\caption{Contoh struktur kode \textit{views}\cite{playforjava}}
	\label{fig:2_views}
\end{figure}

Baris pertama pada kode \textit{views} (gambar \ref{fig:2_views}) digunakan sebagai parameter penerima input dari \textit{models} yang dihubungkan dengan \textit{controllers}. Format deklarasi variabel pada parameter \textit{views} diawali dengan karakter ``@'', lalu ``(namaVariabel$_1$:tipeVariabel$_1$)(namaVariabel$_2$:tipeVariabel$_2$) . . . (namaVariabel$_n$:tipeVariabel$_n$)'', dimana n adalah jumlah parameter yang ingin digunakan pada aplikasi \textit{views}. Variabel pada parameter yang sudah dideklarikan dapat dipanggil dengan menggunakan format ``@namaVariabel'' (seperti dijelaskan pada baris 9 gambar \ref{fig:2_views}).

\subsection{\textit{Controllers}}
\label{sec:controllers}
\textit{Controllers} merupakan bagian pada Play Framework yang terhubung langsung dengan \textit{routes} (subbab \ref{sec:routes}). Jika \textit{action method} yang dikirimkan oleh \textit{routes} sesuai dengan \textit{method} yang dimiliki suatu kelas \textit{controllers}, maka \textit{controllers} akan mengeksekusi fungsi logika yang terdapat pada \textit{method} dan mengembalikan nilai berupa objek dari kelas \textit{Result} (gambar \ref{fig:2_controllers1}). Fungsi dari \textit{controllers} dalam arsitektur MVC adalah sebagai penghubung antara \textit{models} dan \textit{views}. 

\begin{figure}[htbp]
	\centering	
		\includegraphics[scale=0.8]{Gambar/2_controllers1.JPG}
	\caption{Contoh bagaimana hubungan \textit{routes} dan \textit{controllers} dalam memproses HTTP \textit{requests}\cite{playforjava}}
	\label{fig:2_controllers1}
\end{figure}

Penulisan kode pada suatu kelas \textit{controllers} menggunakan bahasa Java dan memiliki aturan khusus (gambar \ref{fig:2_controllers2}). Aturan khusus dijelaskan ke dalam poin-poin sebagai berikut:
\begin{enumerate}
	\item Deklarasi kelas harus \textit{public},
	\item Kelas yang dibuat harus \textit{extends} ``\texttt{play.mvc.Controller}'',
	\item \textit{Method} yang dibuat dalam suatu kelas \textit{controllers} harus bertipe \textit{static},
	\item Nilai kembalian \textit{method} yang dibuat dalam suatu kelas \textit{controllers} harus berupa objek dari kelas Result.
\end{enumerate}

\begin{figure}[htbp]
	\centering	
		\includegraphics[scale=0.8]{Gambar/2_controllers2.JPG}
	\caption{Contoh struktur kode \textit{controllers}\cite{playforjava}}
	\label{fig:2_controllers2}
\end{figure}

