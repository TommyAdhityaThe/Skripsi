\chapter{Dasar Teori}
\label{chap:dasar_teori}

\section{Kode KIRI \textit{Dashboard Server Side}}
\label{sec:dt_kodekiri}

\section{MySQL Spatial Extensions}
\label{sec:mysql_spatial_ex}
	Sarana geografi\cite{mysqlspatial} adalah segala bentuk yang ada di bumi yang memiliki lokasi sebagai penunjuk letak keberadaannya. Geometri adalah teknik yang digunakan untuk mengukur sebuah sarana geografi. Dengan geometri, setiap sarana geografi dapat dinyatakan dengan sebuah titik atau gabungan dari banyak titik (garis). Sarana geografi dapat berupa:
	\begin{enumerate}
		\item Sesuatu yang nyata, contohnya adalah gunung, kolam, kota, dll.
		\item Suatu cakupan ruang yang tidak dapat dilihat, contohnya adalah pembagian batas daerah, cuaca, dll.
		\item Suatu cakupan ruang yang dapat dilihat, contohnya adalah jalan, persimpangan, dll.
	\end{enumerate}
	
	MySQL adalah salah satu perangkat lunak yang digunakan untuk mengatur data-data (\textit{database}) suatu situs web. Bentuk MySQL adalah sekumpulan tabel-tabel yang umumnya memiliki hubungan antar satu dengan yang lainnya. Setiap tabel pada MySQL memiliki kolom dan baris. Kolom pada MySQL menyatakan daftar jenis baris yang ingin dibuat dan baris menyatakan banyaknya data yang ada dalam tabel.
	
	Penamaan suatu kolom dalam MySQL membutuhkan penentuan jenis tipe data yang akan digunakan dalam kolom tersebut. Dalam MySQL terdapat tipe-tipe data yang umum digunakan seperti varchar untuk menyimpan huruf atau kata, int untuk menyimpan angka, boolean untuk menyimpan nilai ``\textit{}true'' atau ``\textit{false}'', dan tipe data lainnya. MySQL Spatial Extensions adalah perluasan dari tipe-tipe data yang disediakan MySQL untuk menyatakan nilai geometri dari suatu sarana geografi. Terdapat 3 tipe data umum dalam MySQL Spatial Extensions, yaitu Point, LineString, dan Polygon.
	
\subsection{Point}
\label{point}
\begin{figure}[htbp]
	\centering
		\includegraphics{D:/Kuliah/Skripsi/SkripsiDesktop/doc/DokumenSkripsi/NEW/Gambar/point.JPG}
	\caption{Contoh penggunaan tipe data Point\cite{mysqltipedata}}
	\label{fig:point}
\end{figure}
Point merupakan salah satu tipe data MySQL Spatial Extensions untuk menyatakan suatu sarana geografi dalam suatu titik yang dinyatakan dalam 2 buah nilai koordinat a dan b dimana a menyatakan letak vertikal dan b menyatakan letak horizontal.

\subsection{LineString}
\label{linestring}
\begin{figure}[htbp]
	\centering
		\includegraphics{D:/Kuliah/Skripsi/SkripsiDesktop/doc/DokumenSkripsi/NEW/Gambar/linestring.JPG}
	\caption{Contoh penggunaan tipe data LineString\cite{mysqltipedata}}
	\label{fig:linestring}
\end{figure}
LineString merupakan salah satu tipe data MySQL Spatial Extensions untuk menyatakan suatu sarana geografi dalam suatu garis yang dinyatakan dalam sekumpulan titik (Point). Garis ditarik berdasarkan urutan dari titik 1 ke titik 2, dari titik 2 ke titik 3, dan seterusnya.

\subsection{Polygon}
\label{Polygon}
\begin{figure}[htbp]
	\centering
		\includegraphics{D:/Kuliah/Skripsi/SkripsiDesktop/doc/DokumenSkripsi/NEW/Gambar/polygon.JPG}
	\caption{Contoh penggunaan tipe data Polygon\cite{mysqltipedata}}
	\label{fig:polygon}
\end{figure}
Polygon merupakan salah satu tipe data MySQL Spatial Extensions untuk menyatakan suatu sarana geografi dalam bentuk ruang 2 dimensi (memiliki luas dan keliling) yang dinyatakan dalam sekumpulan titik (Point). Mirip seperti LineString, garis ditarik berdasarkan urutan dari titik 1 ke titik 2, dari titik 2 ke titik 3, dan seterusnya, lalu dari titik terakhir juga ditarik garis menuju titik 1 sehingga nantinya akan terbentuk ruang 2 dimensi.

	
\section{Play Framework}
\label{sec:play_framework}
Play Framework adalah sekumpulan kerangka kode yang dapat digunakan untuk membangun suatu situs web. Play Framework tidak hanya menggunakan bahasa Java dalam pembuatannya. Bahasa Scala juga digunakan Play Framework dalam beberapa bagian seperti bagian \textit{view} dan \textit{route}\cite{playforjava}. Play Framework menggunakan konsep MVC (\textit{Model} \textit{View} \textit{Controller}) dalam pembuatannya. Konsep MVC pada kode membuat kode mudah dikembangkan baik secara tampilan maupun pengembangan fitur-fiturnya.