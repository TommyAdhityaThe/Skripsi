%_____________________________________________________________________________
%=============================================================================
% data.tex v6 (13-04-2015) \ldots dibuat oleh Lionov - Informatika FTIS UNPAR
%
% Perubahan pada versi 6 (13-04-2015)
% - Perubahan untuk data-data ``template" menjadi lebih generik dan menggunakan
%	tanda << dan >>
%
% Perubahan pada versi sebelumnya
% 	versi 5 (10-11-2013)
% 	- Perbaikan pada memasukkan bab : tidak perlu menuliskan apapun untuk 
%	  memasukkan seluruh bab (bagian V)
% 	- Perbaikan pada memasukkan lampiran : tidak perlu menuliskan apapun untuk
%	  memasukkan seluruh lampiran atau -1 jika tidak memasukkan apapun
%	versi 4 (21-10-2012)
%	- Data dosen dipindah ke dosen.tex agar jika ada perubahan/update data dosen
%   mahasiswa tidak perlu mengubah data.tex
%	- Perubahan pada keterangan dosen	
% 	versi 3 (06-08-2012)
% 	- Perubahan pada beberapa keterangan 
% 	versi 2 (09-07-2012):
% 	- Menambahkan data judul dalam bahasa inggris
% 	- Membuat bagian khusus untuk judul (bagian VIII)
% 	- Perbaikan pada gelar dosen
%_____________________________________________________________________________
%=============================================================================
% 								BAGIAN -
%=============================================================================
% Ini adalah file data (data.tex)
% Masukkan ke dalam file ini, data-data yang diperlukan oleh template ini
% Cara memasukkan data dijelaskan di setiap bagian
% Data yang WAJIB dan HARUS diisi dengan baik dan benar adalah SELURUHNYA !!
% Hilangkan tanda << dan >> jika anda menemukannya
%=============================================================================
%_____________________________________________________________________________
%=============================================================================
% 								BAGIAN I
%=============================================================================
% Tambahkan package2 lain yang anda butuhkan di sini
%=============================================================================
\usepackage{booktabs}
\usepackage[table]{xcolor}
\usepackage{longtable}
\usepackage{amsmath}
%=============================================================================

%_____________________________________________________________________________
%=============================================================================
% 								BAGIAN II
%=============================================================================
% Mode dokumen: menetukan halaman depan dari dokumen, apakah harus mengandung 
% prakata/pernyataan/abstrak dll (termasuk daftar gambar/tabel/isi) ?
% - kosong : tidak ada halaman depan sama sekali (untuk dokumen yang 
%            dipergunakan pada proses bimbingan)
% - cover : cover saja tanpa daftar isi, gambar dan tabel
% - sidang : cover, daftar isi, gambar, tabel (IT: UTS-UAS Seminar 
%			 dan UTS TA)
% - sidang_akhir : mode sidang + abstrak + abstract
% - final : seluruh halaman awal dokumen (untuk cetak final)
% Jika tidak ingin mencetak daftar tabel/gambar (misalkan karena tidak ada 
% isinya), edit manual di baris 439 dan 440 pada file main.tex
%=============================================================================
% \mode{kosong}
% \mode{cover}
% \mode{sidang}
%\mode{sidang_akhir}
\mode{final} 
%=============================================================================

%_____________________________________________________________________________
%=============================================================================
% 								BAGIAN III
%=============================================================================
% Line numbering: penomoran setiap baris, otomatis di-reset setiap berganti
% halaman
% - yes: setiap baris diberi nomor
% - no : baris tidak diberi nomor, otomatis untuk mode final
%=============================================================================
\linenumber{yes}
%=============================================================================

%_____________________________________________________________________________
%=============================================================================
% 								BAGIAN IV
%=============================================================================
% Linespacing: jarak antara baris 
% - single: opsi yang disediakan untuk bimbingan, jika pembimbing tidak
%            keberatan (untuk menghemat kertas)
% - onehalf: default dan wajib (dan otomatis) jika ingin mencetak dokumen
%            final/untuk sidang.
% - double : jarak yang lebih lebar lagi, jika pembimbing berniat memberi 
%            catatan yg banyak di antara baris (dianjurkan untuk bimbingan)
%=============================================================================
\linespacing{single}
% \linespacing{onehalf}
%\linespacing{double}
%=============================================================================

%_____________________________________________________________________________
%=============================================================================
% 								BAGIAN V
%=============================================================================
% Bab yang akan dicetak: isi dengan angka 1,2,3 s.d 9, sehingga bisa digunakan
% untuk mencetak hanya 1 atau beberapa bab saja
% Jika lebih dari 1 bab, pisahkan dengan ',', bab akan dicetak terurut sesuai 
% urutan bab.
% Untuk mencetak seluruh bab, kosongkan parameter (i.e. \bab{ })  
% Catatan: Jika ingin menambahkan bab ke-10 dan seterusnya, harus dilakukan 
% secara manual
%=============================================================================
\bab{ }
%=============================================================================

%_____________________________________________________________________________
%=============================================================================
% 								BAGIAN VI
%=============================================================================
% Lampiran yang akan dicetak: isi dengan huruf A,B,C s.d I, sehingga bisa 
% digunakan untuk mencetak hanya 1 atau beberapa lampiran saja
% Jika lebih dari 1 lampiran, pisahkan dengan ',', lampiran akan dicetak 
% terurut sesuai urutan lampiran
% Jika tidak ingin mencetak lampiran apapun, isi dengan -1 (i.e. \lampiran{-1})
% Untuk mencetak seluruh mapiran, kosongkan parameter (i.e. \lampiran{ })  
% Catatan: Jika ingin menambahkan lampiran ke-J dan seterusnya, harus 
% dilakukan secara manual
%=============================================================================
\lampiran{ }
%=============================================================================

%_____________________________________________________________________________
%=============================================================================
% 								BAGIAN VII
%=============================================================================
% Data diri dan skripsi/tugas akhir
% - namanpm: Nama dan NPM anda, penggunaan huruf besar untuk nama harus benar
%			 dan gunakan 10 digit npm UNPAR, PASTIKAN BAHWA BENAR !!!
%			 (e.g. \namanpm{Jane Doe}{1992710001}
% - judul : Dalam bahasa Indonesia, perhatikan penggunaan huruf besar, judul
%			tidak menggunakan huruf besar seluruhnya !!! 
% - tanggal : isi dengan {tangga}{bulan}{tahun} dalam angka numerik, jangan 
%			  menuliskan kata (e.g. AGUSTUS) dalam isian bulan
%			  Tanggal ini adalah tanggal dimana anda akan melaksanakan sidang 
%			  ujian akhir skripsi/tugas akhir
% - pembimbing: isi dengan pembimbing anda, lihat daftar dosen di file dosen.tex
%				jika pembimbing hanya 1, kosongkan parameter kedua 
%				(e.g. \pembimbing{\JND}{  } ) , \JND adalah kode dosen
% - penguji : isi dengan para penguji anda, lihat daftar dosen di file dosen.tex
%				(e.g. \penguji{\JHD}{\JCD} ) , \JND dan \JCD adalah kode dosen
%
%=============================================================================
\namanpm{Pascal Alfadian Nugroho}{2003730013}	%hilangkan tanda << & >>
\tanggal{19}{3}{2015}			%hilangkan tanda << & >>
\pembimbing{\LNV}{}     %Lihat singkatan pembimbing anda di file dosen.tex
%Lihat singkatan pembimbing anda di file dosen.tex, hilangkan tanda << & >>
\penguji{\TAB}{\CEN} 		%Lihat singkatan penguji anda di file dosen.tex
%Lihat singkatan penguji anda di file dosen.tex, hilangkan tanda << & >>
%=============================================================================

%_____________________________________________________________________________
%=============================================================================
% 								BAGIAN VIII
%=============================================================================
% Judul dan title : judul bhs indonesia dan inggris
% - judulINA: judul dalam bahasa indonesia
% - judulENG: title in english
% PERHATIAN: - langsung mulai setelah '{' awal, jangan mulai menulis di baris 
%			   bawahnya
%			 - Gunakan \texorpdfstring{\\}{} untuk pindah ke baris baru
%			 - Judul TIDAK ditulis dengan menggunakan huruf besar seluruhnya !!
%			 - Gunakan perintah \texorpdfstring{\\}{} untuk baris baru
%=============================================================================

\judulINA{Integrasi Situs Navigasi dengan Situs \textit{Crowdsourcing} Rute Angkot}

\judulENG{Integration of \textit{Angkot} Navigation and Route Crowdsourcing Website}

%_____________________________________________________________________________
%=============================================================================
% 								BAGIAN IX
%=============================================================================
% Abstrak dan abstract : abstrak bhs indonesia dan inggris
% - abstrakINA: abstrak bahasa indonesia
% - abstrakENG: abstract in english
% PERHATIAN: langsung mulai setelah '{' awal, jangan mulai menulis di baris 
%			 bawahnya
%=============================================================================

\abstrakINA{Penelitian ini mengintegrasikan situs navigasi rute angkot
	\url{http://kiri.travel} dengan situs \textit{crowdsourcing} rute angkot 
	\url{https://angkot.web.id}, di mana pengguna dapat berkontribusi memperbaiki
	rute angkot yang salah. Integrasi yang dimaksud adalah sinkronisasi data secara
	berkala dan otomatis, sehingga hasil navigasi yang diberikan mendekati ketepatan
	sesuai di lapangan.

	TODO Sisa abstrak akan dilengkapi saat penelitian berakhir.
}

\abstrakENG{This research integrates \textit{angkot} (public bus) navigation website
	\url{http://kiri.travel} with \textit{angkot} route crowdsourcing website,
	where public user can contribute by fixing erroneus \textit{angkot} routes.
	The aforementioned integration is defined as automatic synchronization of
	both parties' data, such that the result accuracy of navigation is close to
	what found on the field.

	TODO The rest of abstract will be completed at end of research.
} 

%=============================================================================

%_____________________________________________________________________________
%=============================================================================
% 								BAGIAN X
%=============================================================================
% Kata-kata kunci dan keywords : diletakkan di bawah abstrak (ina dan eng)
% - kunciINA: kata-kata kunci dalam bahasa indonesia
% - kunciENG: keywords in english
%=============================================================================
\kunciINA{\textit{crowdsourcing}, angkot, rute, navigasi, \textit{REST}, \textit{JSON}}

\kunciENG{crowdsourcing, route, navigation, REST, JSON}
%=============================================================================

%_____________________________________________________________________________
%=============================================================================
% 								BAGIAN XI
%=============================================================================
% Persembahan : kepada siapa anda mempersembahkan skripsi ini ...
%=============================================================================
\untuk{
	Dipersembahkan untuk seluruh warga Indonesia yang selalu setia menggunakan transportasi umum dan mahasiswa FTIS UNPAR yang akan menyelesaikan skripsinya.
}
%=============================================================================

%_____________________________________________________________________________
%=============================================================================
% 								BAGIAN XII
%=============================================================================
% Kata Pengantar: tempat anda menuliskan kata pengantar dan ucapan terima 
% kasih kepada yang telah membantu anda bla bla bla ....  
%=============================================================================
\prakata{
	Proses bimbingan yang ideal menurut saya membahas hal-hal yang menarik seperti eksplorasi ide, analisis, eksperimen, dll. Kenyataannya, di tengah waktu terbatas, waktu bimbingan terbuang untuk hal-hal yang tidak terlalu penting, seperti tata cara penulisan, \textit{debugging error} di \LaTeX, memperbaiki nalar yang salah, dll. Oleh karena itu, saya membuat contoh skripsi ini, yang terkait dengan template skripsi saya\footnote{\url{https://github.com/pascalalfadian/Skripsi/}}.

	Contoh skripsi ini dibuat untuk memberi gambaran dan panduan bagi mahasiswa dalam pengambilan Skripsi 1 maupun 2. Kode sumber dari dokumen ini dapat diakses di \textit{branch} contoh atau lihat di repositori template skripsi saya\footnote{\url{https://github.com/pascalalfadian/Skripsi/tree/contoh/doc/DokumenSkripsi}}. Jika ada kesalahan ataupun masukan, silahkan berkonsultasi terlebih dahulu pada daftar \textit{issue}\footnote{\url{https://github.com/pascalalfadian/Skripsi/issues}} atau buat \textit{issue} baru dengan label ``contoh''.

	Peringatan: Walaupun segala usaha telah diupayakan untuk kesempurnaan contoh ini, tidak ada jaminan bahwa dokumen ini benar dan ideal di mata penguji.

	Selamat bekerja!
}
%=============================================================================

%_____________________________________________________________________________
%=============================================================================
% 								BAGIAN XIII
%=============================================================================
% Tambahkan hyphen (pemenggalan kata) yang anda butuhkan di sini 
%=============================================================================
\hyphenation{ma-te-ma-ti-ka}
\hyphenation{fi-si-ka}
\hyphenation{tek-nik}
\hyphenation{in-for-ma-ti-ka}
%=============================================================================


%=============================================================================
