\documentclass[a4paper,twoside]{article}
\usepackage[T1]{fontenc}
\usepackage[bahasa]{babel}
\usepackage{graphicx}
\usepackage{graphics}
\usepackage{float}
\usepackage[cm]{fullpage}
\pagestyle{myheadings}
\usepackage{etoolbox}
\usepackage{setspace} 
\usepackage{lipsum} 
\setlength{\headsep}{30pt}
\usepackage[inner=2cm,outer=2.5cm,top=2.5cm,bottom=2cm]{geometry} %margin
% \pagestyle{empty}

\makeatletter
\renewcommand{\@maketitle} {\begin{center} {\LARGE \textbf{ \textsc{\@title}} \par} \bigskip {\large \textbf{\textsc{\@author}} }\end{center} }
\renewcommand{\thispagestyle}[1]{}
\markright{\textbf{\textsc{Laporan Perkembangan Pengerjaan Skripsi\textemdash Sem. Ganjil 2015/2016}}}

\onehalfspacing
 
\begin{document}

\title{\@judultopik}
\author{\nama \textendash \@npm} 

%ISILAH DATA DATA BERIKUT INI:
\newcommand{\nama}{Tommy Adhitya The}
\newcommand{\@npm}{2012730031}
\newcommand{\tanggal}{11/10/2015} %Tanggal pembuatan dokumen
\newcommand{\@judultopik}{\textit{Porting} PHP menjadi Java/Play Framework (Studi Kasus KIRI \textit{Dashboard Server Side})} % Judul/topik anda
\newcommand{\kodetopik}{PAS3901}
\newcommand{\jumpemb}{1} % Jumlah pembimbing, 1 atau 2
\newcommand{\pembA}{Pascal Alfadian, M.Com.}
\newcommand{\pembB}{-}
\newcommand{\semesterPertama}{39 - Ganjil 15/16} % semester pertama kali topik diambil, angka 1 dimulai dari sem Ganjil 96/97
\newcommand{\lamaSkripsi}{1} % Jumlah semester untuk mengerjakan skripsi s.d. dokumen ini dibuat
\newcommand{\kulPertama}{Skripsi 1} % Kuliah dimana topik ini diambil pertama kali
\newcommand{\tipePR}{B} % tipe progress report :
% A : dokumen pendukung untuk pengambilan ke-2 di Skripsi 1
% B : dokumen untuk reviewer pada presentasi dan review Skripsi 1
% C : dokumen pendukung untuk pengambilan ke-2 di Skripsi 2
\maketitle

\pagenumbering{arabic}

\section{Data Skripsi} %TIDAK PERLU MENGUBAH BAGIAN INI !!!
Pembimbing utama/tunggal: {\bf \pembA}\\
Pembimbing pendamping: {\bf \pembB}\\
Kode Topik : {\bf \kodetopik}\\
Topik ini sudah dikerjakan selama : {\bf \lamaSkripsi} semester\\
Pengambilan pertama kali topik ini pada : Semester {\bf \semesterPertama} \\
Pengambilan pertama kali topik ini di kuliah : {\bf \kulPertama} \\
Tipe Laporan : {\bf \tipePR} -
\ifdefstring{\tipePR}{A}{
			Dokumen pendukung untuk {\BF pengambilan ke-2 di Skripsi 1} }
		{
		\ifdefstring{\tipePR}{B} {
				Dokumen untuk reviewer pada presentasi dan {\bf review Skripsi 1}}
			{	Dokumen pendukung untuk {\bf pengambilan ke-2 di Skripsi 2}}
		}

\section{Detail Perkembangan Pengerjaan Skripsi}
Detail bagian pekerjaan skripsi sesuai dengan rencan kerja/laporan perkembangan terkahir :
	\begin{enumerate}
		\item Mempelajari kode situs web KIRI \textit{Dashboard Server Side}(bahasa PHP).\\
		{\bf status :} Ada sejak rencana kerja skripsi.\\
		{\bf hasil :} Sudah melakukan instalasi kode situs web KIRI \textit{Dashboard} (sumber: https://github.com/pascalalfadian/TirtayasaGH) di laptop. Menambahkan \textit{database} ``TirtayasaGH'' ke MySQL yang ada di laptop. Sudah mempelajari bagaimana cara pembuatan \textit{id} dan \textit{password} \textit{administrator} untuk otentikasi KIRI \textit{Dashboard}. Berhasil melakukan otentikasi sebagai seorang \textit{administrator} pada KIRI \textit{Dashboard}. Mencoba fitur CRUD sebagai seorang \textit{administrator} untuk kasus rute angkutan umum dan mencoba fitur \textit{Create}, \textit{Read}, dan \textit{Update} untuk kasus API \textit{keys}. Mempelajari sebagian fungsi-fungsi file .php yang terdapat pada folder ``public\_html\_dev'' dan ``etc''. 
		
		\item Melakukan studi literatur tentang MySQL Spatial Extensions dan Play Framework.\\
		{\bf status :} Ada sejak rencana kerja skripsi.\\
		{\bf hasil :} Sudah melakukan studi literatur mengenai MySQL Spatial Extensions. Mengerti cara menggunakan tipe data \textit{spatial} dalam MySQL, terutama tipe data \textit{LineString} yang digunakan dalam peneletian ini. Sudah melakukan instalasi Play Framework di laptop. Mengerti cara kerja Play Framework secara umum, bagaimana struktur \textit{direktori} Play Framework,  bagaimana \textit{server} Play Framework menerima HTTP \textit{requests} dari pengguna yang diterjemahkan melalui \textit{routes} dan dilanjutkan ke \textit{controllers} yang nantinya akan membalas HTTP \textit{requests} dengan memberikan tampilan berupa \textit{views}. Mengerti dan dapat membuat aplikasi ``Hellow World'' dengan menggunakan Play Framework.

		\item Menganalisis teori-teori untuk membangun KIRI \textit{Dashboard Server Side} dalam bahasa Java dengan menggunakan Play Framework.\\
		{\bf status :} Ada sejak rencana kerja skripsi.\\
		{\bf hasil :} Mengerti penggunaan \textit{controllers}, \textit{routes}, dan \textit{views} (sedikit). Mencoba membuat kode untuk fungsi CRUD (masih gagal).

		\item Merancang KIRI \textit{Dashboard Server Side} dalam bahasa Java dengan menggunakan Play Framework.\\
		{\bf status :} Ada sejak rencana kerja skripsi.\\
		{\bf hasil :} belum ada perkembangan.

		\item Melakukan \textit{porting} kode situs web KIRI \textit{Dashboard Server Side} yang semula dalam bahasa PHP menjadi bahasa Java dengan menggunakan Play Framework.\\
		{\bf status :} Ada sejak rencana kerja skripsi.\\
		{\bf hasil :} belum ada perkembangan.
		
		\item Melakukan pengujian terhadap fitur-fitur yang sudah dibuat\\
		{\bf status :} Ada sejak rencana kerja skripsi.\\
		{\bf hasil :} belum ada perkembangan.

		\item Menulis dokumen skripsi.\\
		{\bf status :} Ada sejak rencana kerja skripsi.\\
		{\bf hasil :} Sudah menulis dokumen skripsi bab 1 dan bab 2.
	\end{enumerate}

\section{Pencapaian Rencana Kerja}
Persentase penyelesaian skripsi sampai dengan dokumen ini dibuat dapat dilihat pada tabel berikut :

\begin{center}
  \begin{tabular}{ | c | c | c | c | l | c |}
    \hline
    1*  & 2*(\%) & 3*(\%) & 4*(\%) &5* &6*(\%)\\ \hline \hline
    1   & 10 & 10 &    &  & 6 \\ \hline
    2   & 10 & 10 &    &  & 8 \\ \hline
    3   & 15 & 15 &    &  & 3 \\ \hline
    4   & 15 &    & 15 &  & 0 \\ \hline
    5   & 15 &    & 15 &  & 0 \\ \hline
    6   & 15 &    & 15 &  & 0 \\ \hline
    7   & 20 & 5  & 15 & {\footnotesize penulisan skripsi hingga bab 3 pada S1} & 3 \\ \hline
    Total  & 100  & 40  & 60 &  & 20\\ \hline
                          \end{tabular}
\end{center}

Keterangan (*)\\
1 : Bagian pengerjaan Skripsi (nomor disesuaikan dengan detail pengerjaan di bagian 5)\\
2 : Persentase total \\
3 : Persentase yang akan diselesaikan di Skripsi 1 \\
4 : Persentase yang akan diselesaikan di Skripsi 2 \\
5 : Penjelasan singkat apa yang dilakukan di S1 (Skripsi 1) atau S2 (skripsi 2)\\
6 : Persentase yang sidah diselesaikan sampai saat ini 

\section{Kendala yang dihadapi}
%TULISKAN BAGIAN INI JIKA DOKUMEN ANDA TIPE A ATAU C
Kendala - kendala yang dihadapi selama mengerjakan skripsi :
\begin{itemize}
	\item Terlalu banyak melakukan prokratinasi
	\item Terlalu banyak godaan berupa hiburan (game, film, dll)
	\item Pembagian waktu dengan mata kuliah lain masih belum baik (banyak tugas-tugas dari mata kuliah lain yang diambil pada semester ini)
\end{itemize}

\vspace{1cm}
\centering Bandung, \tanggal\\
\vspace{2cm} \nama \\ 
\vspace{1cm}

Menyetujui, \\
\ifdefstring{\jumpemb}{2}{
\vspace{1.5cm}
\begin{centering} Menyetujui,\\ \end{centering} \vspace{0.75cm}
\begin{minipage}[b]{0.45\linewidth}
% \centering Bandung, \makebox[0.5cm]{\hrulefill}/\makebox[0.5cm]{\hrulefill}/2013 \\
\vspace{2cm} Nama: \pembA \\ Pembimbing Utama
\end{minipage} \hspace{0.5cm}
\begin{minipage}[b]{0.45\linewidth}
% \centering Bandung, \makebox[0.5cm]{\hrulefill}/\makebox[0.5cm]{\hrulefill}/2013\\
\vspace{2cm} Nama: \pemB \\ Pembimbing Pendamping
\end{minipage}
\vspace{0.5cm}
}{
% \centering Bandung, \makebox[0.5cm]{\hrulefill}/\makebox[0.5cm]{\hrulefill}/2013\\
\vspace{2cm} Nama: \pembA \\ Pembimbing Tunggal
}
`
\end{document}

